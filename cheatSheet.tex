\documentclass[10pt,a4paper,landscape]{article}
\usepackage{multicol}
\usepackage{calc}
\usepackage{ifthen}
\usepackage[landscape]{geometry}
\usepackage{amsmath,amsthm,amsfonts,amssymb}
\usepackage{color,graphicx}
\usepackage{hyperref}
\usepackage{graphicx}
\usepackage{amsmath}
\usepackage{pgfplots}
\pgfplotsset{compat=newest}

\DeclareMathOperator{\sech}{sech}
\DeclareMathOperator{\csch}{csch}
\DeclareMathOperator{\arcsec}{arcsec}
\DeclareMathOperator{\arccot}{arcCot}
\DeclareMathOperator{\arccsc}{arcCsc}
\DeclareMathOperator{\arccosh}{arcCosh}
\DeclareMathOperator{\arcsinh}{arcsinh}
\DeclareMathOperator{\arctanh}{arctanh}
\DeclareMathOperator{\arcsech}{arcsech}
\DeclareMathOperator{\arccsch}{arcCsch}
\DeclareMathOperator{\arccoth}{arcCoth} 


\pdfinfo{
	/Title (Calculus 2 Cheat Sheet.pdf)
	/Creator (TeX)
	/Producer (pdfTeX 1.40.0)
	/Author (Ahmed ELtokhy)
	/Subject (Cheat Sheet)
	/Keywords (pdflatex, latex,pdftex,tex)}

\ifthenelse{\lengthtest { \paperwidth = 297mm}}
{ \geometry{top=.15in,left=.15in,right=.15in,bottom=.15in} }
{\ifthenelse{ \lengthtest{ \paperwidth = 297mm}}
{\geometry{top=1cm,left=1cm,right=1cm,bottom=1cm} }
{\geometry{top=1cm,left=1cm,right=1cm,bottom=1cm} }
}

% Turn off header and footer
\pagestyle{empty}

% Redefine section commands to use less space
\makeatletter
\renewcommand{\section}{\@startsection{section}{1}{0mm}%
{-1ex plus -.5ex minus -.2ex}%
{0.5ex plus .2ex}%
{\normalfont\large\bfseries}}
\renewcommand{\subsection}{\@startsection{subsection}{2}{0mm}%
{-1ex plus -.5ex minus -.2ex}%
{0.5ex plus .2ex}%
{\normalfont\normalsize\bfseries}}
\renewcommand{\subsubsection}{\@startsection{subsubsection}{3}{0mm}%
{-1ex plus -.5ex minus -.2ex}%
{.1ex plus .2ex}%
{\normalfont\small\bfseries}}
\makeatother


\setcounter{secnumdepth}{0}


\setlength{\parindent}{0pt}
\setlength{\parskip}{0pt plus 0.5ex}

\newtheorem{example}[section]{Example}

\begin{document}
\raggedright
\footnotesize
\begin{multicols*}{5}


	\setlength{\premulticols}{1pt}
	\setlength{\postmulticols}{1pt}
	\setlength{\multicolsep}{1pt}
	\setlength{\columnsep}{2pt}

	\section{Derivatives}
	\scriptsize
	$\frac{d}{dx} e^x=e^x$\newline
	$\frac{d}{dx} \sin(x)=\cos(x)$\newline
	$\frac{d}{dx} \cos(x)=-\sin(x)$\newline
	$\frac{d}{dx} \tan(x)=\sec^2(x)$\newline
	$\frac{d}{dx} \cot(x)=-\csc^2(x)$\newline
	$\frac{d}{dx} \sec(x)=\sec(x)\tan(x)$\newline
	$\frac{d}{dx} \csc(x)=-\csc(x)\cot(x)$\newline
	$\frac{d}{dx} \sin^{-1}=\frac{1}{\sqrt{1-x^2}}, x \in [-1,1]$\newline
	$\frac{d}{dx} \cos^{-1}=\frac{-1}{\sqrt{1-x^2}}, x \in [-1,1]$\newline
	$\frac{d}{dx} \tan^{-1}=\frac{1}{1+x^2}, \frac{-\pi}{2}\le x \le \frac{\pi}{2}$\newline
	$\frac{d}{dx} \sec^{-1}=\frac{1}{\mid x \mid \sqrt{x^2-1}}, |x| > 1$\newline
	$\frac{d}{dx} \sinh(x)=\cosh(x)$\newline
	$\frac{d}{dx} \cosh(x)=\sinh(x)$\newline
	$\frac{d}{dx} \tanh(x)=sech^2(x)$\newline
	$\frac{d}{dx} \coth(x)=-csch^2(x)$\newline
	$\frac{d}{dx} sech(x)=-sech(x)\tanh(x)$\newline
	$\frac{d}{dx} csch(x)=-csch(x)\coth(x)$\newline
	$\frac{d}{dx} \sinh^{-1}=\frac{1}{\sqrt{x^2+1}}$\newline
	$\frac{d}{dx} \cosh^{-1}=\frac{-1}{\sqrt{x^2-1}}, x > 1$\newline
	$\frac{d}{dx} \tanh^{-1}=\frac{1}{1-x^2} -1 < x < 1$\newline
	$\frac{d}{dx} sech^{-1}=\frac{1}{x \sqrt{1-x^2}}, 0 < x < 1$
	$\frac{d}{dx} \ln(x) = \frac{1}{x} $\newline
	$\frac{d}{dx} b^x = b^x \ln{x} $ 

	\section{Integrals}
	\scriptsize
	$\int \sin(x) dx = -\cos(x) $\newline
	$\int \cos(x) dx = \sin(x) $\newline
	$\int \tan(x) dx = -\ln|\cos(x)| $\newline
	$\int \sec x \tan x dx = \sec x $\newline
	$\int \sec x dx = \ln | \sec x + \tan x |$\newline 
	$\int \sec^2(x) dx = \tan(x) + C $ 
	$\int \sec^3(x) dx= \frac{1}{2}(\sec x \tan x + \ln| \sec x \tan x |)  $ 
	$\int \csc^2(x) dx= -\cot(x)$\newline
	$\int \csc (x) \cot (x) dx = - \csc(x)$\newline
	$\int \csc (x) dx = \ln | \csc (x) - \cot (x) | $\newline
	$\int \cot(x) dx = \ln|\sin(x)| $\newline
	$\int \frac{1}{x}dx = \ln|x|$\newline
	$\int e^x dx = e^x $\newline
	$\int a^x dx = \frac{1}{\ln a} a^x $\newline
	$\int e^{ax} dx = \frac{1}{a} e^{ax} $\newline
	$\int \frac{1}{\sqrt{a^2-x^2}} dx = \sin^{-1}(\frac{x}{a}) $\newline
	$\int \frac{1}{a^2+x^2} dx = \frac{1}{a} \tan^{-1}(\frac{x}{a}) $\newline
	$\int \frac{1}{x^2-a^2} dx = \frac{1}{2a} \ln |\frac{x-a}{x+a}|$\newline 
	$\int \frac{1}{\sqrt{x^2 \pm a^2}} dx=\ln |x+\sqrt{x^2 \pm a^2}|$\newline
	$\int \frac{1}{x\sqrt{x^2-1}} dx = \sec^{-1}(x) $\newline
	$\int \sinh(x) dx = \cosh(x) $\newline
	$\int \cosh(x) dx = \sinh(x) $\newline
	$\int \tanh(x) dx = \ln|\cosh(x)| $\newline
	$\int \tanh(x)sech(x) dx = -sech(x) $\newline
	$\int \sech^2(x) dx = \tanh(x) $\newline
	$\int \csch(x)\coth(x) dx = -\csch(x) $\newline
	$ \int \ln(x) dx = (x \ln(x))-x $\newline
	$ \int^r_{-r} \sqrt{ r^2 - x^2 }\ dx = \frac{ \pi r^2 }{ 2 }  $\newline
	$\int \frac{1}{\sqrt{x}} dx = 2\sqrt{x} $\newline
	Integrals of inverse trig functions are solved using integration by parts.\newline

	\textbf{Weierstrass} \newline
	$ t = \tan{ \frac{x}{2} } \quad \cos{ x } = \frac{ 1-t^2 }{ 1+t^2 }\newline \sin{ x } = \frac{2t}{1+t^2} \quad dx = \frac{2}{1+t^2}dt  $\newline

$\int \sin^m{ x } \cos^n{ x } $\newline
 $ m $ is odd factor sines, then sub $ u= \cos{ x }  $
If $ n $ is odd factor cosines, then sub $ u= \sin{ x }  $. 

If both even, use $ \frac{ 1- \cos{ (2x) }  }{ 2 }, \frac{ 1+ \cos(2x) }{ 2 }  $.
If both odd, factor one with less power.\newline

$\int \tan^mx \sec^nx$

$n$ even save one $ \sec^2x $ and sub $ u = \tan{ x }  $. 

$m$ odd save one $ \sec x \tan{ x } $ and sub $ u = \sec x $ \newline

$\int \sqrt{ a^2-x^2 } dx \implies x = a \sin{ \theta }$
$\int \sqrt{ a^2+x^2 } dx \implies x = a \tan{ \theta }$
$\int \sqrt{ x^2-a^2 } dx \implies x = a \sec{\theta}$ 

$ \int^{\infty}_{1} \frac{1}{x^p} ,\ p>1 : \text{converge, } p 	\leq 1 : \text{diverges} $


	\textbf{Integration by Parts}\newline
	$ \int u dv = uv-\int v du $

	\section{Functions/ Identities}
	\textbf{Most of trig identities work with hyperbolic, exceptions below}\newline  
	$\sin(\cos^{-1}(x)) = \sqrt{1-x^2}$\newline
	$\cos(\sin^{-1}(x)) = \sqrt{1-x^2} $\newline
	$\sec(\tan^{-1}(x)) = \sqrt{1+x^2} $\newline
	$\tan(\sec^{-1}(x)) \newline = (\sqrt{x^2-1} \: $if$ \: x \ge 1)\newline =(-\sqrt{x^2-1} \: if \: x \le -1)$
	$\sinh^{-1}(x) = \ln{x+\sqrt{x^2+1}} $\newline
	$\sinh^{-1}(x) = \ln{x+\sqrt{x^2-1}}, \: x \ge -1 $\newline
	$\tanh^{-1}(x) = \frac{1}{2}\ln{x+\frac{1+x}{1-x}}, \: 1 < x < -1 $\newline
	$\sech^{-1}(x) = \ln[{\frac{1+\sqrt{1-x^2}}{x}}], \: 0 < x \le -1 $\newline
	$\sinh(x) = \frac{e^{x}-e^{-x}}{2} $\newline
	$\cosh(x) = \frac{e^{x}+e^{-x}}{2} $\newline
	$ \sin^2(x)+\cos^2(x) = 1 $\newline
	$ 1+\tan^2(x) = \sec^2(x) $\newline
	$ 1+\cot^2(x) = \csc^2(x) $\newline
	$ \sin(x\pm y) = \sin(x)\cos(y)\pm\cos(x)\sin(y) $\newline
	$ \cos(x\pm y) = \cos(x)\cos(y)\pm\sin(x)\sin(y) $\newline
	$ \tan(x\pm y) = \frac{\tan(x)\pm\tan(y)}{1 \mp \tan(x)\tan(y)} $\newline
	$ \sin(2x) = 2\sin(x)\cos(x) $\newline
	$ \cos(2x) = \cos^{2}(x) - \sin^{2}(x) $\newline
	$ \cosh(n^{2}x)-\sinh^{2}x = 1 $\newline
	$ 1+\tan^2(x) = \sec^2(x) $\newline
	$ 1+\cot^2(x) = \csc^2(x) $\newline
	$ \sin^2(x) = \frac{1-\cos(2x)}{2} $\newline
	$ \cos^2(x) = \frac{1+\cos(2x)}{2} $\newline
	$ \tan^2(x) = \frac{1-\cos(2x)}{1+\cos(2x)} $\newline
	$ \sin(-x) = -\sin(x) $\newline
	$ \cos(-x) = \cos(x) $\newline
	$ \tan(-x) = -\tan(x) $\newline
	$ \cosh^2(x) - \sinh^2(x) = 1 $\newline
	$ 1-\tanh^2(x) = \sech^2(x) $
	$ \sin{ A } \cos{ B } = \frac{1}{2} ( \sin{( A-B )} + \sin{( A+B )}   ) $
	$ \sin{ A } \sin{ B } = \frac{1}{2} ( \cos{( A-B )} - \cos{( A+B }   )) $
	$ \cos{ A } \cos{ B } = \frac{1}{2} ( \cos{ (A-B) } + \cos{ (A+B) }   ) $
	\subsection{3D}
	given two points:\newline
	$ (x_1, y_1, z_1) $ and $ (x_2, y_2, z_2)$,\newline
	Distance between them:\newline
	$ \sqrt{(x_1-x_2)^2+(y_1-y_2)^2+(z_1-z_2)^2} $\newline
	Midpoint:\newline
	$ (\frac{x_1+x_2}{2},\frac{y_1+y_2}{2},\frac{z_1+z_2}{2}) $\newline
	Sphere with center (h,k,l) and radius r:\newline
	$ (x-h)^2 + (y-k)^2 + (z-l)^2 = r^2 $

	\subsection{Vectors}
	Vector: $ \vec{u} $\newline
	Unit Vector: $ \hat{u} $\newline
	Magnitude: $ ||\vec{u}|| = \sqrt{u_1^2+u_2^2+u_3^2}$\newline
	Unit Vector: $ \hat{u} = \frac{\vec{u}}{||\vec{u}||} $\newline

	\textbf{Dot Product}\newline
	$ \vec{u} \cdot \vec{v} $\newline
	Produces a Scalar \newline
	(Geometrically, the dot product is a vector projection)\newline
	$\vec{u} = < u_1, u_2, u_3 >$\newline
	$\vec{v} = < v_1, v_2, v_3 >$\newline
	$ \vec{u} \cdot \vec{v} = \vec{0} $ means the two vectors are Perpendicular
	$ \theta $ is the angle between them.\newline
	$\vec{u} \cdot \vec{v} = ||\vec{u}||\:||\vec{v}||\cos(\theta) $\newline
	$\vec{u} \cdot \vec{v} = u_1v_1 + u_2v_2 + u_3v_3 $\newline
	NOTE:\newline
	$ \hat{u} \cdot \hat{v} = \cos(\theta) $\newline
	$ ||\vec{u}||^2 = \vec{u} \cdot \vec{u} $\newline
	$ \vec{u} \cdot \vec{v} = 0 $ when $ \bot $\newline
	Angle Between $ \vec{u} $ and $ \vec{v} $:\newline
	$ \theta = \cos^{-1}(\frac{\vec{u} \cdot \vec{v}}{||\vec{u}||\:||\vec{v}||}) $\newline
	Projection of $ \vec{u} $ onto $ \vec{v} $:\newline
	$ pr_{\vec{v}}\vec{u} = (\frac{\vec{u} \cdot \vec{v}}{||\vec{v}||^2})\vec{v} $\newline

	\textbf{Cross Product}\newline
	$\vec{u} \times \vec{v}$\newline
	Produces a Vector\newline
	(Geometrically, the cross product is the area of a paralellogram with sides $ ||\vec{u}|| $ and $ ||\vec{v}|| $)\newline
	$\vec{u} = < u_1, u_2, u_3 >$\newline
	$\vec{v} = < v_1, v_2, v_3 >$\newline
	\[
		\vec{u} \times \vec{v} = 
		\begin{vmatrix}
			\hat{i} & \hat{j} & \hat{k} \\
			u_1 & u_2 & u_3 \\
			v_1 & v_2 & v_3
		\end{vmatrix}
	\]\newline 
	$ \vec{u} \times \vec{v} = \vec{0} $ means the vectors are parallel \newline
	$ \vec{ a } \times ( \vec{ b } \times \vec{ c } )= ( \vec{ a } . \vec{ c } ) \vec{ b } - ( \vec{ a } . \vec{ b } ) \vec{ c }$ \newline

	\textbf{Volume of Parallelpiped} \newline 
	\begin{equation*}
		( \vec{ v } \times \vec{ u } ) . \vec{ w }
	\end{equation*}

	However, in the case the question gave you three vectors directly
	\begin{equation*}
		Volume = det \left( \left [
			\begin{matrix}
				v_1&v_2&v_3\\u_1&u_2&u_3\\w_1&w_2&w_3
			\end{matrix}
			\right]
		\right)
	\end{equation*}


	\subsection {Lines and Planes}
	\textbf{Equation of a Plane}\newline
	$ (x_0, y_0, z_0) $ is a point on the plane and $ <A,B,C> $ is a normal vector

	$A(x-x_0)+B(y-y_0)+C(z-z_0) = 0$\newline
	$ <A,B,C> \cdot <x-x_0, y-y_0, z-z_0> = 0 $\newline
	$ Ax+By+Cz = D $ where $ D=Ax_0+By_0+Cz_0 $\newline

	\textbf{Equation of a line}\newline
	A line requires a Direction Vector $ \vec{u}=<u_1,u_2,u_3> $ and a point $(x_1,y_1,z_1)$\newline

	Parametric equation: 
	$ x=u_1t+x_1 \quad y=u_2t+y_1 \quad z=u_3t+z_1 $

	Symmetric equations:
	$	t= \frac{ x-x_1 }{u_1} = \frac{ y-y_1 }{u_2} = \frac{ z-z_1 }{u_3}) $



	\textbf{Distance from a Point to a Plane}\newline
	The distance from a point $(x_0,y_0,z_0)$ to a plane Ax+By+Cz=D can be expressed by the formula:\newline
	$ d=\frac{|Ax_0+By_0+Cz_0-D|}{\sqrt{A^2+B^2+C^2}} $\newline

	\textbf{A line intersects a plane} Just put the equations of $x,y,z$ in the equation of the plane given, and solve for $t$. Then you'll have the points. 
	\newline 

	\textbf{Distance between two skew lines through $ P_1P_2,\ P_3P_4 $ } \newline
	\[
		D = \frac{ Vol_{parallelepiped} }{ Area_{base} } = \frac{ | ( \vec{ P_1P_2 } \times \vec{ P_3P_4 } ) . \vec{ P_1P_3 } |}{ | \vec{ P_1P_2 } \times  \vec{ P_3P_4 } | } 
	\]

	Or go for the intuitive approach of forcing perpendicularity.\newline

	\textbf{If two vectors are parallel} $| \vec{ v_1 } \times  \vec{ v_2 } |=0 $\newline 

	\textbf{Distance between a point and a line} $ \vec{ v } $  \newline
	$ D = \frac{| \vec{ Pv_0 } \times \ \vec{ v } | }{ | \vec{ v } |  }$ \newline

	\subsection {Other Information}
	$ \frac{\sqrt{a}}{\sqrt{b}} = \sqrt{\frac{a}{b}} $\newline
	Law of Cosines:\newline
	$ a^2 = b^2 + c^2 - 2bc(\cos(\theta)) $\newline
	Quadratic Formula:\newline
	$ \frac{ -b \pm \sqrt{ b^2-4ac } }{ 2a } $, when $ ax^2+bx+c=0 $ 
	\section{Integration by parts usage}
\newline
$\int u dv = uv - \int vdu $ \newline
\subsection{Case 1 }
\newline
The integration is the product of 2 functions: 1 easy to integrate and the other vanishes by differentiation.
use Table method: \newline
\includegraphics[width=4cm, height=2cm]{Tabular-Method-Integration-by-Parts-2}
\newline
\subsection{Case 2}
\newline
The presence of ln, log, inverse trig, inverse hyp, or any weird function without its derivative
\newline put u = weird function and dv = the other function(or dx) \newline 
\subsection{Case 3}
\newline $\int e^ax * sin bx || cos bx dx  $ \newline
N.B (does not work with sinh and cosh as they can be factorized) \newline
you can put any of the functions as u or dv but remain with the same choice till the end. Integrate till you reach the same integration then solve as an algebraic equation.\newline
\subsection{Case 4: The reduction formula }
\newline In = solution + In-m \newline
\section{Trigonometric Integrals}
\subsection{Sin and Cos: }
\int sin^n x cos^m x dx
\newline \newline
\text{I- n is odd:  use } u = cos x \newline
\text{II- m is odd: use }u = sin x \newline
\text{III- m and n are odd: use any u} \newline
\text{IV- m and n are even: use double angle rule} \newline
		cos^2 x = 1/2(1+cos2x) \newline 		sin^2 x = 1/2(1-cos2x) \newline
N.B (\text{use }cos^2 x + sin^2 x = 1)
\newline 
\subsection{tan and sec:}
\int tan^n x sec^m x dx \newline
\text{I- n is odd:  use }u = sec x \newline
\text{II- m is even: use }u = tan x \newline
\text{III- m is even and n is odd: use any u} \newline
\text{IV- m is odd and n is even: } \newline \text{use integration by parts Case 3.} \newline
N.B (\text{use }tan^2 x + 1 = sec^2 x) \newline
\text{- It works with }\int cot^n x csc^m x dx \newline \text{ using }(cot^2 x + 1 = csc^2 x) \newline
\text{- Sometimes non of the above work} \newline \text{and you should compose it to sin and cos.} \newline
\section{Trigonometric substitution}
\begin{center}
\begin{tabular}{ |c|c|c|c| } 
 \hline
 Expression & Substitution & Identity \\ 
 a^2 - x^2 & x = a sin\theta dx = a cos\theta d\theta & cos^2 x + sin^2 x = 1 \\ 
  a^2 + x^2 & x = a tan\theta dx = a sec^2 \theta d\theta & tan^2 x + 1 = sec^2 x \\
 x^2 - a^2 & x = a sec\theta dx = a sec \theta tan \theta d\theta &tan^2 x + 1 = sec^2 x \\
 \hline
\end{tabular}
\end{center}
\section{Integration by partial fraction}
- A rational function $\frac{p(x)}{q(x)}$ \newline
- proper (degree of p(x) $<$ q(x)) if improper use long division \newline
- q(x) can be factorized \newline
to get A, B and C you can cover the denominator of each one alone and substitute the x with its root. Or unify the denominator and substitute any three numbers in x. \newline
\subsection{Case 1}
q(x) is a product of distinct linear functions. \newline
Ex: $\int \frac{p(x)}{(x-a)(x+b)(x-c)} dx$ = $\frac{A}{x-a}+\frac{B}{x+b} +\frac{C}{x-c}$ \newline
\subsection{Case 2}
q(x) is a product of linear functions but one of them is repeated. \newline
Ex: $\int \frac{p(x)}{(x+a)(x-b)^2} dx = \frac{A}{x+b} + \frac{B}{(x-c)^2} + \frac{C}{(x-c)}$\newline
\subsection{Case 3}
q(x) contain an irreducible and not repeated quadratic function. \newline
use a polynomial with a degree less then q(x) by 1 \newline
Ex: $\int \frac{p(x)}{(x-a)(x^2+e^2)} dx = \frac{A}{x-a} + \frac{Bx + C}{x^2 + e^2}$\newline
use $\int \frac{1}{x^2 + a^2} dx = \frac{1}{a} arctan(\frac{x}{a}) + C$\newline
\subsection{Case 4}
Rationalization: when you have p(x) is a root take u = p(x) and proceed. \newline
\section{Volume}
\subsection{Disk Method}
volume of any revolving graph = $\pi \int^b_{a} f(x)^2 dx$\newline where f(x) represents the radius of the disk and a,b are the end points. (When revolving about x the integral is in terms of x and same for the y)
\subsection{Washer Method}
Volume of a graph between two functions = $\pi \int^b_{a} f(x)^2 - g(x)^2 dx$ \newline where f(x) represents the outer radius, g(x) the inner radius and a,b are the end points. 
\newline \newline \newline \newline \newline\newline \newline \newline \newline \newline\newline \newline
\subsection{Shell Method}
(When revolving about y the integral is in terms of x and vice versa) \newline V = $ 2\pi \int^b_{a} x f(x) dx$ If two graphs then f(x) will be the difference between the inner and the outer functions.
\section{Area}
$\int^b_{a} f(x) - g(x) dx$ Where f(x) is the larger function (further from the axis). \newline
\section{Improper integration}
	\newpage

\end{multicols*}
\end{document}
